\documentclass[a4paper,11pt]{letter}

\usepackage[left=2cm,right=2cm,top=2cm,bottom=2cm]{geometry}

\signature{Dr. Denis Demidov}
\address{
Joint Supercomputer Center \\
Russian Academy of Sciences \\
Lobachevsky st. 2/31\\
420111 Kazan, Russia \\
email: ddemidov@ksu.ru
}
\begin{document}

\begin{letter}{
    Hans Petter Langtangen\\
    Editor-in-Chief\\
    SIAM Journal of Scientific Computing
    }
\opening{Dear Dr. Langtangen,}

On behalf of all the authors, I would like to submit revised version of
manuscript entitled ``Programming CUDA and OpenCL: a Case Study Using Modern
C++ Libraries'' for publication in the ``Software and High-Performance
Computing'' section of SIAM Journal of Scientific Computing.

There were several remarks made by the referees, which I would like to address
below.

\begin{enumerate}
    \item The first point made by Referee \#1 is that a reference
        implementation in CUDA or OpenCL is not presented. This is result of
        unclear wording from our side, since we intended to use Thrust library
        as a reference implementation. Thrust is a widely accepted GPGPU
        library which is supported by the major vendor (NVIDIA) and is included
        into the CUDA Toolkit distribution package. Hence, we believe the
        library is suitable candidate for the reference implementation. We
        rephrased the relevant part of the introduction to make this point more
        clear to the readers.
    \item The other point the Referee makes is that ViennaCL implementation of
        the phase oscillator chain example used manually coded OpenCL kernel,
        which, by his opinion, undermined the idea of using libraries instead
        of hand coded low-level code. The authors of ViennaCL have since
        improved their library thus eliminating use of any hand coded kernels
        in our paper. We have changed the related code snippets to reflect the
        change.
    \item Probably the most important objection from the first Referee is that
        the result we show lack performance portability, especially when run on
        a CPU. We agree that it is hard to achieve performance portability with
        current state of OpenCL and hardware differences, but we believe that
        this issue should be addressed by library authors and not by
        application developers. To emphasize the point, authors of
        VexCL and ViennaCL % and hopefully CMTL
        have added several CPU-related optimizations to their codes.
        This resulted in much improved agreement between performance results of
        the libraries. We have updated the relevant plots, Table~4.1, and parts
        of the discussion section.
    \item Referee \#2 suggested, in addition to timings for the different
        problems, to report what fractions of the available computational
        resource are being exploited. We believe this is very helpful addition
        and we have accordingly updated Table~4.1.
\end{enumerate}

We would like to thank both Referees for the fruitful discussion. We hope that
the paper in its revised form would be suitable for publication in the SIAM
Journal of Scientific Computing.


The co-authors of the paper are: Dr. Karsten Ahnert (karsten.ahnert@gmx.de,
Ambrosys GmbH, Geschw.-Scholl-Str.63a, 14471 Potsdam, Germany), Dr. Karl Rupp
(rupp@mcs.anl.gov, Mathematics and Computer Science Division, Argonne National
Laboratory, 9700 South Cass Avenue, Argonne, IL 60439, USA), and Dr. Peter
Gottschling (Peter.Gottschling@tu-dresden.de, Technische Universit\"at Dresden,
Institut f\"ur Wissenschaftliches Rechnen, 01062 Dresden, Germany).

\closing{Sincerely,}

\end{letter}

\end{document}
Dear Dr. Langtangen,

On behalf of all the authors
