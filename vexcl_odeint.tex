\documentclass[1p]{elsarticle}

\usepackage{amsmath}
\usepackage{amssymb}
\usepackage{graphicx}
\usepackage[colorlinks,unicode,linkcolor=black]{hyperref}

\journal{Journal of Parallel and Distributed Computing}

\begin{document}

\begin{frontmatter}

\title{Comparing OpenCL and CUDA: a Case Study\\ Using Modern C++ Libraries
odeint, VexCL, and Thrust}

\author{Karsten Ahnert}
\ead{kahnert@uni-potsdam.de}
\address{
Institut f\"ur Physik und Astronomie, Universit\"at Potsdam,\\
Karl-Liebknecht-Strasse 24/25, 14476 Potsdam-Golm, Germany
}

\author{Denis Demidov}
\ead{ddemidov@ksu.ru}
\address{
Kazan Branch of Joint Supercomputer Center,
Russian Academy of Sciences,\\
Lobachevsky st. 2/31, 420008 Kazan, Russia
}

\begin{abstract}
    We present comparison of OpenCL and CUDA frameworks performance. The
    comparison is based on odeint~--- modern C++ library for solution of
    ordinary differential equations. Odeint is designed in a very flexible way
    such that the algorithms are completely independent from the underlying
    containers and even from the basic algebraic computations. This allows to
    effectively use VexCL or Thrust libraries to solve ODEs with OpenCL or CUDA
    technologies. We found that OpenCL and CUDA work equally well for problems
    of large sizes, although OpenCL is not well suited for smaller problems due
    to its higher initialization overhead.
\end{abstract}

\begin{keyword}
    GPGPU \sep OpenCL \sep CUDA \sep performance comparison
\end{keyword}

\end{frontmatter}

\section{Introduction}

Recently, GPGPU based computing has aquiered considerable momentum in
scientific community. This is confirmed by constantly increasing number of GPU
based supercomputers in
top500\footnote{\href{http://top500.org/}{http://top500.org/}} list. Major
programming frameworks are NVIDIA CUDA and OpenCL.  The former is proprietary
parallel computing architecture developed by Nvidia for general purpose
computing on Nvidia graphics cards, and the latter is open, royalty-free
standard for cross-platform, parallel programming of modern processors backed
by Khronos group.

\section{Libraries used in the case study}
\subsection{odeint}
\subsection{VexCL}
\subsection{Thrust}

\section{Performance evaluation}

\section{Conclusion}

\section{Acknowledgments}

This work has been supported by RFBR grant No 12-07-0007.

\nocite{*}
\bibliographystyle{model1-num-names}
\bibliography{ref}

\end{document}
