\documentclass{letter}

\signature{Dr. Denis Demidov}
\address{
Kazan branch of Joint Supercomputer Center,\\
Russian Academy of Sciences, \\
Lobachevsky st. 2/31,\\
420011 Kazan, Russia 
}
\begin{document}

\begin{letter}{SIAM Journal of Scientific Computing,\\Editor}
\opening{Dear Sir/Madam,}

Enclosed is a paper entitled ``Programming CUDA and OpenCL: a Case Study Using
Modern C++ Libraries''. Please accept it as a candidate for publication in the
``Software and High-Performance Computing'' section of SIAM Journal of
Scientific Computing.

The co-authors of the paper are:
\begin{itemize}
    \item Dr. Denis Demidov\footnote{Corresponding author}, Kazan branch of
	Joint Supercomputer Center, Russian Academy of Sciences, Lobachevsky
	st. 2/31, 420011 Kazan, Russia.\\
	email: ddemidov@ksu.ru.
    \item Dr. Karsten Ahnert, Institut f\"ur Physik und Astronomie,
	Universit\"at Potsdam, Karl-Liebknecht-Strasse 24/25, 14476
	Potsdam-Golm, Germany.\\ email: karsten.ahnert@gmx.de.
    \item Dr. Karl Rupp, Mathematics and Computer Science Division, Argonne
	National Laboratory, 9700 South Cass Avenue, Argonne, IL 60439, USA. \\
	email: rupp@mcs.anl.gov.
    \item Dr. Peter Gottschling, Technische Universit\"at Dresden, Institut
	f\"ur Wissenschaftliches Rechnen, 01062 Dresden, Germany.\\
	email: Peter.Gottschling@tu-dresden.de.
\end{itemize}

In the paper we present a comparison of several modern C++ libraries providing
high-level interfaces for programming multi- and many-core architectures on top
of CUDA or OpenCL.  The comparison focuses on the solution of ordinary
differential equations and is based on odeint, a framework for the solution of
ordinary differential equations. Odeint is designed in a very flexible way and
may be easily adapted for effective use of libraries such as Thrust, VexCL, or
ViennaCL, using CUDA or OpenCL technologies.  We found that CUDA and OpenCL
work equally well for problems of large sizes, while OpenCL has higher overhead
for smaller problems.  Furthermore, we show that modern high-level libraries
allow to effectively use the computational resources of many-core GPUs or
multi-core CPUs without much knowledge of the underlying technologies. We
believe the contribution of this study warrants its publication in the SIAM
Journal of Scientific Computing.

This paper is our original unpublished work and it has not been submitted to
any other journal for reviews.

\closing{Sincerely,}

\end{letter}

\end{document}
